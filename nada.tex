\documentclass{sig-alternate}

\usepackage{graphicx}
\usepackage{subfig}
\usepackage{url}
\usepackage{verbatim}
\usepackage{xspace}

\usepackage{watermark}

\DeclareCaptionType{copyrightbox}
%\usepackage{etoolbox}
%\makeatletter
%\patchcmd{\maketitle}{\@copyrightspace}{}{}{}
%\makeatother

\newcommand{\eg}{e.g.\xspace}
\newcommand{\ie}{i.e.\xspace}
\newcommand{\etc}{etc.\xspace}
\newcommand{\etal}{et al.\xspace}
\newcommand{\vect}[1]{\ensuremath{\overrightarrow{#1}}}
\newcommand{\tm}[1]{\emph{#1}}

\newcommand{\namesep}{\hspace{1cm}}
\toappear{}
%\renewcommand{\baselinestretch}{0.937}

\special{! /pdfmark where
  {pop} {userdict /pdfmark /cleartomark load put} ifelse [
    /Author ()
    /Title (Flight Patterns: Visualizing Air Traffic Patterns over the United States)
    /Keywords ()
    /DOCINFO pdfmark}


\title{NormStad Flight Analysis: Visualizing Air Traffic Patterns over the United States
}%\vspace{-20pt}}


\numberofauthors{1}
\author{
\alignauthor
\begin{tabular}{c@{\namesep}c@{\namesep}c}
 Samet Ayhan & Brendan C. Fruin & Fan Yang
\end{tabular}\\
\affaddr{Department of Computer Science, University of Maryland}\\
\affaddr{College Park, MD  20742 USA}
\email{\{sayhan, brendan, fyang\}@cs.umd.edu}
}

\newcommand{\refname}{Bibliography}

\begin{document}

\maketitle

\begin{abstract}

In this paper, we describe the NormSTAD Flight Analysis system, a novel visualization application that enables
interactive visualization of air traffic patterns using Aircraft
Situations Display to Industry (ASDI) data. A web-based visualization
is demonstrated which allows users to analyze flight data and make 
discoveries pertaining to their 4D trajectories which include 
their time, distance, altitude, and speed. Unique patterns discovered
in this application could result in less fuel consumption and more efficient
management of departure and arrivals by air traffic controllers.

The application consumes ASDI data then normalizes flight attributes
including distance, speed, altitude, and time along the flight path. This 
information is then displayed on a line chart which can be customized
through filtering, coloring and selection. Attributes pertaining to 
selected flights can be viewed in a details on demand fashion. The result
is a both intuitive and visually appealling visualization with the goals
of revealing flight paths, spotting trends and revealing outliers.

\end{abstract}

\category{H.3.3}{Information Storage and Retrieval}{Information Search and Retrieval}[Information filtering]
\category{\\H.3.5}{Information Storage and Retrieval}{Online Information Services}[Web-based services]
\category{\\H.5.2}{Information Interfaces and Presentation}{User Interfaces}

\vspace{-2mm}
\terms{Design, Human Factors, Performance}



\section{Introduction}
\label{sec-introduction}

% BCF 11/28 - cite NAS
The National Airspace System (NAS) is a complex non-deterministic system that
is impacted continually by both major and minor variables including aircraft
delays and human decisions that largely cannot be accurately forecasted.
The system has developed to offer feedback and response at all levels from
gate agents to the Command Center with the intent of restoring the desired
efficient state. The result is a self-ordering system that is broadly
similar, but has different daily operations. An important point to note is that
a seemingly insignificant event such as
a delay in obtaining a wheel chair can have large impacts in delays
as \emph{slots} are missed and reassigned. At every stage decisions are being
made to recover the system and keep it as close to optimal given its current
state. However, there is currently no method of quantifying the effects of 
a decision or comparing them to an alternate decision. NAS operations
are recorded concurrently across different systems in different formats.
If this information was collated and catalogued, it would be possible
to analyze the NAS operations to identify inefficiencies, disruptive events 
and poor decisions along with the resulting impacts on airspace users.

In 1992 the Federal Aviation Administration (FAA), started a program to provide
real-time flight plan and track information for the NAS to airlines and other
oganizations. The feed known as Aircraft Situation Display to Industry (ASDI) is a 
product of the Enhanced Traffic Management System (ETMS). It originates from
the Traffic Flow Management (TFM) Production Center located at the William
J. Hughes Technical Center in Atlantic City, New Jersey. Figure~\ref{table1} shows the 
number of ASDI messages for a single random day while Figure~\ref{table2} shows the number
of supporting records for the main message set.


A novel visualization system is needed to aggregate this data in order for 
users to detect anomalies and discover unique patterns. To be able to compare 
flights of varying distance or time, the data values are normalized to 
create a standardized display of time series allowing direct comparison between
flights. 

\newcommand{\incfig}[2]{\includegraphics[#2]{figs/#1}\label{#1}}

\begin{figure}
\centering
\subfloat[]{\incfig{table1}{width=1\columnwidth}}\hfill
\subfloat[]{\incfig{table2}{width=1\columnwidth}}
\caption{
\subref{table1} Caption for Figure 1a -- SAMET 
 \subref{table2} Caption for Figure 1b -- SAMET
}
\label{two-tables}
\end{figure}

The rest of this paper is organized as follows. Section~\ref{sec-related-work}
discusses related
work. Section~\ref{sec-asdi} explains the ASDI data and its attributes. 
Section~\ref{sec-architecture} describes
the overall architecture and the interface is described in Section~\ref{sec-interface}.
Section~\ref{sec-expert-review}
provides the expert reviews
and feedbacks of the implementation. Section~\ref{sec-future-work}
discusses future work, while our conclusions
are outlined in Section~\ref{sec-conclusion}.

\section{Related Work}
\label{sec-related-work}

There has been a great amount of research in flight data pertaining to algorithms
for optimal trajectories, anomaly detection and conflict 
resolution~\cite{Basu09, Cao06, Chu10, Liu08, Rama06, Wang04}.
Liu et al.~\cite{Liu08} study departure and arrival delays and how these delays
can propagate to future flight delays and cancellations. Our system instead focuses
on in-air flight data to help study why these delays may be happening with 
respect to time of day, flight number, airline or flight trajectory. 
Chu et al.~\cite{Chu10} similarly
attempt to detect anomalies in aircraft cruise data by using time, location, altitude 
and speed. They also normalized their data as was done in the NormSTAD system. However,
the NormSTAD system utilizes only historical data while their system only studied the results
of a simulation. While the study of such algorithms for detection of anomalies and conflict
resolution is vitally important to the field of aviation, the goal of the NormSTAD Flight Analysis 
Tool is to allow easy and timely analysis of large amounts of flight data without the need
for such algorithms.

Landry~\cite{Landry11} and Khoury et al.~\cite{Khoury06} stress the importance
of informative visualizations for the study of air traffic control systems.
Landry 



\section{ASDI Data}
\label{sec-asdi}

In this section, we introduce ASDI data and its
attributes, overall. In addition, data normalization
process is described.

ASDI subsystem of the Traffic Flow
Management System (TFMS) allows near
real-time air traffic data to be disseminated to
members of the aviation industry. The data stream
is made available through the U.S. Department
of Transportation's Volpe Transportation Center.
The data stream consists of data elements which
show the position and flight plans of all aircraft
in U.S. Elements include the location, altitude,
airspeed, destination, estimated time of arrival and
tail number or designated identifier of air carrier
and general aviation aircraft operating on IFR flight
plans within U.S. airspace.

Due to the project’s requirements, only a
subset of historical ASDI data was used. The data
set included Delta and Delta Connection, Atlantic
Southeast Airlines’ flights departing from Seattle
Tacoma Airport (KSEA) for the duration of July
and August 2012.

Flights pertaining to Delta Airlines consisted
of the following flight numbers:
• DAL842, departing from Seattle Tacoma
and arriving to New York JFK
• DAL1043, departing from Seattle Tacoma
and arriving to New York JFK Airport
• DAL2410, departing from Seattle Tacoma
and arriving to Detroit Metropolitan
Wayne County Airport

Flights pertaining to Delta Connection,
Atlantic Southeast Airlines consisted of the
following flight numbers:
• ASA24, departing from Seattle Tacoma
and arriving to Boston Logan International
Airport
• ASA678, departing from Seattle Tacoma
and arriving to Denver International
Airport

Dataset was generated by merging a subset of
various message types including:
• Departure Information
• Track Information and
• Flight Management Information

The process yielded the following fields:
• Source Date
• Source Time
• Aircraft Id
• Flight Key
• Speed
• Altitude Type
• Altitude Format
• Altitude
• Latitude
• Longitude

In addition, certain values were normalized
so that various flights or time series of same
flights could be overlaid on the same line chart.
Normalized values for interactive visualization
included:
• Altitude
• Speed
• Distance
• Planned Flight Time and
• Actual Flight Time

All normalized values range between 0 and 1
with an exception that Actual Flight Time may
well go over 1 or remain under 1 due to fact
that non on-time flights usually take less or
more than Planned Flight Time.

\section{Architecture}
\label{sec-architecture}

\section{Interface}
\label{sec-interface}

The NormSTAD Flight Analysis user interface initially shows four panels of display as seen in 
FIG labeled Filters, Line Chart, Map and Details on Demand. Each panel allows for 
direct interaction or display of the data and can be minimized to
give more room to the other panels by selecting 
the arrow in the upper right corner of the specific panel. 

When the application is first opened, the Line Chart panel of NormSTAD Flight Analysis is shown in the top
center of the browser displaying two graphs with all flights displayed for the 
loaded dataset. A noticeable issue is that lines are often close together and hard to
distinguish.
 To combat this issue, the bottom graph allows for selection of a range of 
the horizontal axis. By selecting a range along the horizontal axis, the coordinate
system of the top graph is redrawn to have only values contained within the selected subset
creating a zoom-in effect. This range can be expanded or contracted and even moved by clicking,
holding, and dragging the range window. A mouse hover or mouse click on a
line turns the line red and 
thicker making it easier to see. A mouse click on a line has the effect of
updating the Details on Demand
to display flight information and removing this information on the second click of the line. 
If the "Show Flight Path" checkbox is checked in the Filters panel, then a mouse click on a line also displays
the flight path in the Map panel on a Google Maps map for the sampled latitude and longitude points along with a label containg the airport code at the departure and arrival airports.
The Details on Demand panel displays the flight number, the departure and arrival airports, the date,
the arrival time and the duration of the flight in minutes for the flights selected in the
Line Graph.

The Filters panel allows the user to limit the results that they see on the Line
Chart. The first section, "Choose Airlines", is for selection of airlines to display 
in the Line Chart which is set to all airlines by default.
Upon selection of an airline, the Line Chart is updated accordingly and the
"Choose Flights" section is populated with all
flights for the selected airline(s). Note that a user can select multiple adjacent
rows for both the Choose Airlines and the Choose Flights sections
by holding the Shift key and selecting rows or 
multiple rows by holding the Control key (Apple key on Apple computers) and selecting rows. 
Selection of row(s) in the Choose Flights section allows for updating of the Line Chart
by the flight number. The flights can be further filtered by limiting the flight data
using the "Start Date" and "End Date" drop down calendar menus. The data displayed 
in the Line Chart can be changed by changing the "Line chart type" value which subsequently
changes the axis or both of the axes where 
all values are normalized. By default, this value is set to "Distance vs. Actual Flight Time",
but it can be changed to "Distance vs. Planned Flight Time", "Altitude vs. Actual Flight Time",
or "Speed vs. Actual Flight Time". The lines can be colored by their airline or by their 
flight number using the "Color by" drop-down menu. 


\section{Expert Reviews and Feedback}
\label{sec-expert-review}

\section{Future Work}
\label{sec-future-work}

\section{Conclusion}
\label{sec-conclusion}

\section{Acknowledgements}
\label{sec-acknowledgements}
We would like to thank all of the
participants who helped us evaluate alpha and
beta versions of the tool. We were able to make
improvements based on their expert reviews,
and feedback. We would especially like to thank
Professor Michael Ball with his valuable guidance
on determining a niche area in air traffic control
and management to work on and overcoming issues
pertaining to design and implementation of Flight
Pattern Analysis tool. We can’t thank enough to
Professor Ben Shneiderman for his advice that
motivated us and kept us focused along the way.

\bibliographystyle{abbrv}
\begin{small}
\bibliography{nada}
\end{small}

\end{document}

