\documentclass{sig-alternate}

\usepackage{graphicx}
\usepackage{subfig}
\usepackage{url}
\usepackage{verbatim}
\usepackage{xspace}

\usepackage{watermark}

\DeclareCaptionType{copyrightbox}
%\usepackage{etoolbox}
%\makeatletter
%\patchcmd{\maketitle}{\@copyrightspace}{}{}{}
%\makeatother

\newcommand{\eg}{e.g.\xspace}
\newcommand{\ie}{i.e.\xspace}
\newcommand{\etc}{etc.\xspace}
\newcommand{\etal}{et al.\xspace}
\newcommand{\vect}[1]{\ensuremath{\overrightarrow{#1}}}
\newcommand{\tm}[1]{\emph{#1}}

\newcommand{\namesep}{\hspace{1cm}}
\toappear{}
%\renewcommand{\baselinestretch}{0.937}

\special{! /pdfmark where
  {pop} {userdict /pdfmark /cleartomark load put} ifelse [
    /Author ()
    /Title (Flight Patterns: Visualizing Air Traffic Patterns over the United States)
    /Keywords ()
    /DOCINFO pdfmark}


\title{Flight Patterns: Visualizing Air Traffic Patterns over the United States
}%\vspace{-20pt}}


\numberofauthors{1}
\author{
\alignauthor
\begin{tabular}{c@{\namesep}c@{\namesep}c}
 Samet Ayhan & Brendan C. Fruin & Fan Yang
\end{tabular}\\
\affaddr{Department of Computer Science, University of Maryland}\\
\affaddr{College Park, MD  20742 USA}
\email{\{sayhan, brendan, fyang\}@cs.umd.edu}
}

\newcommand{\refname}{Bibliography}

\begin{document}

\maketitle

\begin{abstract}

In this paper, we describe a novel visualization application that enables
interactive visualization of air traffic patterns using Aircraft
Situations Display to Industry data. A web-based visualization
is demonstrated which allows users to analyze flight data and make 
discoveries pertaining to its 4D trajectories. Unique patterns discovered
in this application could result in less fuel consumption and more efficient
management of departure and arrivals by air traffic controllers.

The application consumes ASDI data then normalizes flight attributes
including distance, speed, altitude, and time along the flight path. This 
information is then displayed on a line chart which can be customized
through filtering, coloring and selection. Attributes pertaining to 
selected flights can be viewed in a details on demand fashion. The result
is a both intuitive and visually appealling visualization with the goals
of revealing flight paths, spotting trends and revealing outliers.

\end{abstract}

\category{H.3.3}{Information Storage and Retrieval}{Information Search and Retrieval}[Information filtering]
\category{H.3.5}{Information Storage and Retrieval}{Online Information Services}[Web-based services]
\category{H.5.2}{Information Interfaces and Presentation}{User Interfaces}

\vspace{-2mm}
\terms{Design, Human Factors, Performance}



\section{Introduction}
\label{sec-introduction}

% BCF 11/28 - cite NAS
The National Airspace System (NAS) is a complex non-deterministic system that
is impacted continually by both major and minor variables including aircraft
delays and human decisions that largely cannot be accurately forecasted.
The system has developed to offer feedback and response at all levels from
gate agents to the Command Center with the intent of restoring the desired
efficient state. The result is a self-ordering system that is broadly
similar, but has different daily operations. An important point to note is that
a seemingly insignificant event such as
a delay in obtaining a wheel chair can have large impacts in delays
as \emph{slots} are missed and reassigned. At every stage decisions are being
made to recover the system and keep it as close to optimal given its current
state. However, there is currently no method of quantifying the effects of 
a decision or comparing them to an alternate decision. NAS operations
are recorded concurrently across different systems in different formats.
If this information was collated and catalogued, it would be possible
to analyze the NAS operations to identify inefficiencies, disruptive events 
and poor decisions along with the resulting impacts on airspace users.

In 1992 the Federal Aviation Administration (FAA), started a program to provide
real-time flight plan and track information for the NAS to airlines and other
oganizations. The feed known as Aircraft Situation Display to Industry (ASDI) is a 
product of the Enhanced Traffic Management System (ETMS). It originates from
the Traffic Flow Management (TFM) Production Center located at the William
J. Hughes Technical Center in Atlantic City, New Jersey. Figure 1a shows the 
number of ASDI messages for a single random day while Figure 1b shows the number
of supporting records for the main message set.

A novel visualization system is needed to aggregate this data in order for 
users to detect anomalies and discover unique patterns. To be able to compare 
flights of varying distance or time, the data values are normalized to 
create a standardized display allowing direct comparison.

The rest of this paper is organized as follows. Section~\ref{sec-related-work}
discusses related
work. Section~\ref{sec-asdi} explains the ASDI data and its attributes. 
Section~\ref{sec-arch} describes
the overall architecture and the interface and Section~\ref{sec-expert-review}
provides the expert reviews
and feedbacks of the implementation. Section~\ref{sec-future-work}
discusses future work, while our conclusions
are outlined in Section~\ref{sec-conclusion}.

\section{Related Work}
\label{sec-related-work}


\section{ASDI}
\label{sec-asdi}

\section{Architecture and Interface}
\label{sec-arch}


\section{Expert Reviews and Feedback}
\label{sec-expert-review}

\section{Future Work}
\label{sec-future-work}

\section{Conclusion}
\label{sec-conclusion}


\bibliographystyle{abbrv}
\begin{small}
\bibliography{nada}
\end{small}

\end{document}

